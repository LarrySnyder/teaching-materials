% Slides by Lawrence V. Snyder, Lehigh University

%%%%%%%%%%%%%%%%%%%%%%%%%%%%%%%%%%%%%%%%%%%%%%%%%%%%%%%%%%%%%%%%%%%%%%%%%%%%%%%%%%%%%%%%%%%%%%%%%
%
% Headers and macros for slides for ISE 240 (Intro to Operations Research)
% Based on Hiller and Lieberman textbook, "Introduction to Operations Research" 
% Slides developed by Prof. Larry Snyder, Lehigh University
%
% Please do not distribute LaTeX source code for slides without prior permission from the author
% Contact him at larry.snyder@lehigh.edu or http://coral.ie.lehigh.edu/~larry
%
%%%%%%%%%%%%%%%%%%%%%%%%%%%%%%%%%%%%%%%%%%%%%%%%%%%%%%%%%%%%%%%%%%%%%%%%%%%%%%%%%%%%%%%%%%%%%%%%%


\documentclass{beamer} 
%\documentclass[handout]{beamer}

\usepackage{pgf,pgfarrows,pgfnodes,pgfautomata,pgfheaps,pgfshade}
\usepackage{amsmath,amssymb}
\usepackage[latin1]{inputenc}
\usepackage{colortbl}
\usepackage[english]{babel}
\usepackage[normalem]{ulem} % for strikethrough font

\usepackage{tikz}
\usetikzlibrary{snakes}
\usetikzlibrary{decorations}
\usetikzlibrary{calc}
\usepgflibrary{shapes.geometric}
\usetikzlibrary{intersections}
\usetikzlibrary{positioning}

\usepackage{pgfplots}
\usepgfplotslibrary{fillbetween}


\usepackage{graphicx}
\usepackage[breakable]{tcolorbox}
\usepackage{alltt}

\usepackage{inconsolata} % fixed-width font for tt

% listings package
\usepackage{listings}
\lstset{% set up for AMPL
	basicstyle=\tiny\ttfamily,
	keywordstyle=\color{purple},
	commentstyle=\color{ForestGreen},
	backgroundcolor=\color{blue!5},
	frame=lines,
	morecomment=[l]{\#},
	morecomment=[l][\color{BrickRed}]{<--}, % not really a comment, just for pointing things out on the slides
	morekeywords={param, var, set, maximize, minimize, sum, subject, subj, to, binary, model, data, solve, union, inter, diff, symdiff, cross, reset, option, display, include, printf, let, for, if, then, else},
	sensitive=true,
	tabsize=4,
}
\newcommand{\myinline}[1]{\lstinline[basicstyle=\small\ttfamily]!#1!}

\usepackage{beamerthemetree} 			% choose beamer theme

\input{dvipsnam.def}				% color definitions

\beamertemplatenavigationsymbolsempty		% turn off navigation symbols

\usefoottemplate{\tinycolouredline{white}{\insertshorttitle\hfill\insertframenumber}}
						% insert footer: date and page number

\beamerboxesdeclarecolorscheme{allblue}{structure!75!black}{structure!75!black}
\beamerboxesdeclarecolorscheme{twotone}{structure!75!black}{structure!10!averagebackgroundcolor}
						% define color schemes for beamer boxes
						
\setcounter{MaxMatrixCols}{20}			% reset maximum # of columns in matrix environment


% internal comments -- to suppress, comment out first line below and un-comment second line
%\newcommand{\comment}[1]{\textcolor{red}{\bf [#1]}}
%\newcommand{\comment}[1]{}


% environment for code examples
\newenvironment{mycode}{\vspace{0em}\begin{tcolorbox}[breakable,boxrule=0.25mm]\begin{alltt}}{\end{alltt}\end{tcolorbox}\vspace{0em}}



% environments
\newcommand{\beq}{\begin{equation*}}
\newcommand{\eeq}{\end{equation*}}
\newcommand{\bit}{\begin{itemize}}
\newcommand{\eit}{\end{itemize}}
\newcommand{\ben}{\begin{enumerate}}
\newcommand{\een}{\end{enumerate}}

% LP text
\newcommand{\minimize}{\text{minimize}}
\newcommand{\maximize}{\text{maximize}}
\newcommand{\st}{\text{subject to}}
\newcommand{\shortmin}{\text{min}}
\newcommand{\shortmax}{\text{max}}
\newcommand{\shortst}{\text{s.t.}}
\newcommand{\primal}{\ensuremath{(\text{P})}}
\newcommand{\dual}{\ensuremath{(\text{D})}}

% math shortcuts
\newcommand{\smfrac}[2]{\textstyle{\frac{#1}{#2}}}
\newcommand{\artvar}[1]{\bar{x}_{#1}}
\newcommand{\matname}[1]{\mathbf{#1}}
\newcommand{\mymat}[1]{\left[\begin{matrix} #1 \end{matrix}\right]}
\newcommand{\matA}{\matname{A}}
\newcommand{\matAinv}{\matA^{-1}}
\newcommand{\matx}{\matname{x}}
\newcommand{\matb}{\matname{b}}
\newcommand{\matI}{\matname{I}}
\newcommand{\maty}{\matname{y}}
\newcommand{\matnull}{\matname{0}}
\newcommand{\matc}{\matname{c}}
\newcommand{\matB}{\matname{B}}
\newcommand{\matBinv}{\matB^{-1}}
\newcommand{\pijn}[1]{p_{ij}^{(#1)}}
\newcommand{\matP}{\matname{P}}
\newcommand{\matPn}[1]{\matP^{(#1)}}
\newcommand{\liminfty}{\lim_{n\rightarrow\infty}}

% variance
\newcommand{\var}{\text{Var}}
\newcommand{\cov}{\text{Cov}}

% font styles
\newcommand{\vocab}[1]{\textcolor{Blue}{\bf #1}}	% for vocabulary words
\newcommand{\basic}[1]{\textcolor{Red}{#1}}		% for basic variables

% Beamer boxes
\newcommand{\boxtext}[1]{\begin{beamerboxesrounded}[scheme=allblue]{}{\textcolor{White}{#1}}\end{beamerboxesrounded}}
\newcommand{\boxtexttitle}[2]{\begin{beamerboxesrounded}[scheme=twotone]{\textcolor{White}{#1}}{#2}\end{beamerboxesrounded}}
		% include headers and macros

% TITLE PAGE

\title{AMPL Tutorial}
\author{Prof.~Larry Snyder \\ Lehigh University}
\date{1/22/20}


\begin{document}

%\lstset{language=Pascal}

\frame[plain]{\titlepage}

% OUTLINE SLIDE

\section<presentation>*{Outline}

\frame
{
  \frametitle{Outline}
  \tableofcontents[hidesubsections]
}

\AtBeginSection[]
{
  \frame
  {
    \frametitle{Outline}
    \tableofcontents[current,hidesubsections]
  }
}


%%%%%%%%%%%%%%%%%%%%%%%%%%%%
% Introduction
%%%%%%%%%%%%%%%%%%%%%%%%%%%%

\section{Introduction}

\frame
{
	\frametitle{AMPL's Role}
	
	\begin{columns}
	
	\column[t]{1in}
	
	We write optimization models like this:

	\vspace{-1em}
	
	{\scriptsize
	\begin{align*}
	\text{maximize} \quad & \sum_{j=1}^n p_jx_j \\
	\text{subject to} \quad & \sum_{j=1}^n a_{ij}x_j \le b_i \quad \forall i \\
		& x_j \le d_j\delta_j \quad \forall j \\
		& x_j \ge 0 \quad \forall j \\
		& \delta_j \in \{0,1\} \quad \forall j
	\end{align*}
	}

	\column[t]{1in}
	
	\onslide<3-> {
	AMPL does the translating:
	
	\vspace{3em}
	
	\begin{center}
	\begin{tikzpicture}
	
	\node[draw=black,fill=none] (ampl) {\includegraphics[width=0.5in]{AMPL-logo.jpg}};
	\draw[->] ($ (ampl.west) +(-0.25in,0) $) -- (ampl.west);
	\draw[->] (ampl.east) -- ($ (ampl.east) +(0.25in,0) $);
	
	\end{tikzpicture}
	\end{center}
	}
	
	\column[t]{1in}
	
	\onslide<2-> {
	The solver (algorithm) needs them to look like this:
	
	{\tiny
	\begin{align*}
	c & = [2.5, 2.0, 1.7, 0, 0, 0] \\
	A & = \left[\begin{matrix} 1 & 2 & 1 & 0 & 0 & 0 \\
						0 & 1 & 3 & 0 & 0 & 0 \\
						1 & 0 & 0 & -30 & 0 & 0 \\
						0 & 1 & 0 & 0 & -10 & 0 \\
						0 & 0 & 1 & 0 & 0 & -15 \end{matrix}\right] \\
	b & = \left[\begin{matrix} 15 & 20 & 0 & 0 & 0 \end{matrix} \right]'
	\end{align*}
	}
	}

	\end{columns}
	
}


\begin{frame}[fragile] % to allow code listings in slides -- see https://tex.stackexchange.com/a/130131/63847

	\frametitle{AMPL Blends Algebraic Notation and Computer Code}
	
	\begin{lstlisting}
param num_resources;					# number of resources (M)
param num_products;						# number of products (N)

set RESOURCES := 1..num_resources;		# set of resources
set PRODUCTS := 1..num_products;		# set of products

param fixed_cost {PRODUCTS};			# fixed cost (K_j)
param var_cost {PRODUCTS};				# variable cost (C_j)
param sales_price {PRODUCTS};			# sales price (p_j)
param sales_potential {PRODUCTS};		# sales potential (d_j)
param avail {RESOURCES};				# resource availability (b_i)
param needs {RESOURCES, PRODUCTS};		# num units of i needed for j (a_ij)

var ProduceAmt {PRODUCTS} >= 0;			# num units of product to produce (x_j)
var Produce {PRODUCTS} binary;			# produce product? (delta_j)

maximize Profit:
	sum {j in PRODUCTS} sales_price[j] * ProduceAmt[j] 
	- sum {j in PRODUCTS} (fixed_cost[j] * Produce[j] 
		+ var_cost[j] * ProduceAmt[j]);
	
subject to Supply {i in RESOURCES}:
	sum {j in PRODUCTS} needs[i,j] * Produce[j] <= avail[i];
	
subject to LinkingAndSalesPotential {j in PRODUCTS}:
	Produce[j] <= sales_potential[j] * Produce[j];
	\end{lstlisting}

\end{frame}


\begin{frame}[fragile]

	\frametitle{Some Basics}
	
	\bit
	\item AMPL is case-sensitive
	\item AMPL ignores whitespace
	\item Your model will go into a text file (\myinline{.mod} and/or \myinline{.dat})
	\item You will type commands like \myinline{solve} at the \myinline{ampl:} prompt
	\item Comments are denoted by a number sign (\myinline{\#}) 
	\item Every line must end with a  semicolon (\myinline{;})
	\eit

\end{frame}


%%%%%%%%%%%%%%%%%%%%%
% CENTRE COUNTY: EXPLICIT
%%%%%%%%%%%%%%%%%%%%%

\section{The Centre County Problem: Explicit Form}

\begin{frame}[fragile]

	\frametitle{Centre County Problem}

	\bit
	\item Centre County, PA is considering  4 potential community development projects:
	
		\vspace{0.5em}
		\centerline{\footnotesize
		\begin{tabular}{lccc}
		Project & Daily Usage & Cost & Land Space (acres) \\
		\hline
		Park & 600 & \$50,000 & 8 \\
		Basketball court & 100 & \$20,000 & 0 \\
		Recreation center & 300 & \$150,000 & 4 \\
		Swimming pool & 500 & \$70,000 & 5 \\
		\hline
		\end{tabular}
		}
		\vspace{0.5em}
	
	\item Basketball court will be built in park
		\bit
		\item No space needed
		\item But cannot build unless build park
		\eit
	\eit
\end{frame}
	
\begin{frame}[fragile]

	\frametitle{Centre County Problem}
	
	\bit
	\item \$200,000 from state
	\item 15 acres available
	\item Goal: select projects to max daily usage s.t.\ budget and land constraints	
	\eit
	
	\vspace{1em}
	
	{\small {\em Source}: Ravindran, Griffin, and Prabhu, {\em Service Systems Engineering and Management}, CRC Press, 2018.}
	
\end{frame}

	

\begin{frame}[fragile]

	\frametitle{Centre County Problem}
	
	\footnotesize
	\beq \setlength{\arraycolsep}{0.15em} 
	\begin{array}{lrcrcrcrcl@{\qquad}c}
		\shortmax & Z = 600x_1 & + & 100x_2 & + & 300x_3 & + & 500x_4 & & & \\
		\shortst & 50x_1 & + & 20x_2 & + & 150x_3 & + & 70x_4 & \le & 200 & \\
		& 8x_1 & & & + & 4x_3 & + & 5x_4 & \le & 15 & \\
		 & x_1 & , & x_2 & , & x_3 & , & x_4 & \le & 1 \\
		 & x_1 & , & x_2 & , & x_3 & , & x_4 & \ge & 0 
	\end{array} 
	\eeq
	\normalsize
	
	\bit
	\item (For now, we'll allow the decision variables to be continuous and ignore the ``if basketball, then park'' constraint)
	\item Open AMPL and create a file called \myinline{centre.mod}
	\item Type the following:
	\eit
	
\begin{lstlisting}
var x1 >= 0, <= 1;
var x2 >= 0, <= 1;
var x3 >= 0, <= 1;
var x4 >= 0, <= 1;

maximize Profit: 600 * x1 + 100 * x2 + 300 * x3 + 500 * x4;

subject to Budget: 50 * x1 + 20 * x2 + 150 * x3 + 70 * x4 <= 200;

subject to Space: 8 * x1 + 4 * x3 + 5 * x4 <= 15;
\end{lstlisting}

\end{frame}

\begin{frame}[fragile]

	\frametitle{Centre County Problem}
		
\begin{lstlisting}
var x1 >= 0, <= 1;
var x2 >= 0, <= 1;
var x3 >= 0, <= 1;
var x4 >= 0, <= 1;

maximize Profit: 600 * x1 + 100 * x2 + 300 * x3 + 500 * x4;

subject to Budget: 50 * x1 + 20 * x2 + 150 * x3 + 70 * x4 <= 200;

subject to Space: 8 * x1 + 4 * x3 + 5 * x4 <= 15;
\end{lstlisting}

	\bit
	\item Notice that:
		\bit
		\item Every line ends with a semicolon
		\item Decision variables are declared using \myinline{var}
		\item The objective function is declared using \myinline{maximize} (or \myinline{minimize})
		\item Constraints are declared using \myinline{subject to} (or shorten to \myinline{subj to})
		\item The objective function and constraints each get a name (\myinline{Profit}, \myinline{Budget}, \myinline{Space}); names must be unique
		\eit
	\eit
	
\end{frame}

\begin{frame}[fragile]

	\frametitle{Centre County Problem} 

\begin{lstlisting}
var x1 >= 0, <= 1;
var x2 >= 0, <= 1;
var x3 >= 0, <= 1;
var x4 >= 0, <= 1;

maximize Profit: 600 * x1 + 100 * x2 + 300 * x3 + 500 * x4;

subject to Budget: 50 * x1 + 20 * x2 + 150 * x3 + 70 * x4 <= 200;

subject to Space: 8 * x1 + 4 * x3 + 5 * x4 <= 15;
\end{lstlisting}

	\bit
	\item At the \myinline{ampl:} prompt, type:
	\eit
	
\begin{lstlisting}
ampl: model centre.mod;			<-- tells AMPL which model you want to solve
ampl: option solver cplex;		<-- tells AMPL which solver you want to use
ampl: solve;					<-- tells AMPL to solve it!
\end{lstlisting}

	\bit
	\item You should see:
	\eit
	
\begin{lstlisting}
CPLEX 12.8.0.0: optimal solution; objective 1320
1 dual simplex iterations (0 in phase I)
\end{lstlisting}

\end{frame}

\begin{frame}[fragile]

	\frametitle{Congrats, You Just Solved Your First AMPL Model!}
	
\begin{lstlisting}
CPLEX 12.8.0.0: optimal solution; objective 1320
1 dual simplex iterations (0 in phase I)
\end{lstlisting}

	\bit
	\item \myinline{objective 1320} means the optimal objective function value is 1320
	\item Let's find out the values of the decision variables
	\item We ask AMPL for values using the \myinline{display} command:
	\eit

\begin{lstlisting}
ampl: display x1, x2, x3, x4;
x1 = 1
x2 = 1
x3 = 0.4
x4 = 1
\end{lstlisting}

\end{frame}

\begin{frame}[fragile]

	\frametitle{Binary Variables}
	
	\bit
	\item Now let's make the variables binary instead of continuous
	\item Replace \myinline{>= 0, <= 1} with \myinline{binary}:
	\eit
	
\begin{lstlisting}
var x1 binary;
var x2 binary;
var x3 binary;
var x4 binary;
\end{lstlisting}

	\bit
	\item Let's also add the ``if basketball, then park'' constraint:
	\eit
	
\begin{lstlisting}
subject to IfBasketballThenPark: x2 <= x1;
\end{lstlisting}

\end{frame}

\begin{frame}[fragile]

	\frametitle{Binary Variables}
	
	\bit
	\item When we change the \myinline{.mod} file, we must tell AMPL to \myinline{reset}
	\item Then re-load the model
	\item Then re-solve
	\item (We don't need to specify the solver again)
	\eit

\begin{lstlisting}
ampl: reset;
ampl: model centre.mod;
ampl: solve;
CPLEX 12.8.0.0: optimal integer solution; objective 1200
0 MIP simplex iterations
0 branch-and-bound nodes
\end{lstlisting}

\end{frame}

\begin{frame}[fragile]

	\frametitle{Binary Variables}
	
\begin{lstlisting}
ampl: reset;
ampl: model centre.mod;
ampl: solve;
CPLEX 12.8.0.0: optimal integer solution; objective 1200
0 MIP simplex iterations
0 branch-and-bound nodes
\end{lstlisting}

	\bit
	\item Again, let's find out the values of the variables:
	\eit
	
\begin{lstlisting}
ampl: display x1, x2, x3, x4;
x1 = 1
x2 = 1
x3 = 0
x4 = 1
\end{lstlisting}

	\bit
	\item Yay, they're binary!
	\item (By the way, you can use the $\uparrow$ and $\downarrow$ keys to scroll through earlier commands)
	\eit

\end{frame}


%%%%%%%%%%%%%%%%%%%%%
% CENTRE COUNTY: ALGEBRAIC
%%%%%%%%%%%%%%%%%%%%%

\section{The Centre County Problem: General Algebraic Form}

\begin{frame}[fragile]

	\frametitle{A More General Algebraic Approach}
	
	\bit
	\item So far, so good
	\item But: This format would be a big pain if lots of variables
	\item The solution is to use {\bf sets} to index our {\bf parameters}, {\bf decision variables}, {\bf summations}, and {\bf constraints}
	\item Let's omit the ``if basketball, then park'' constraint for now
	\item Create a new file called \myinline{centrecounty.mod} that contains:
	\eit

\lstinputlisting[lastline=13]{centrecounty-slides.mod}

\end{frame}


\begin{frame}[fragile]

	\frametitle{A More General Algebraic Approach}
	
\lstinputlisting[lastline=13]{centrecounty-slides.mod}

	\bit
	\item Notice that:
		\bit
		\item Parameters and variables are indexed by the set; this is indicated with curly braces: \myinline{param usage \{PROJECTS\}}
		\item In objective and constraints, use square brackets to index the parameters and variables: \myinline{usage[p]}
		\item Text after the comment symbol (\myinline{\#}) is ignored
		\item Summation indices are also indicated with curly braces: \myinline{sum \{j in PROJECTS\}}
		\eit
	\eit

\end{frame}

\begin{frame}[fragile]

	\frametitle{But Wait---There Are No Numbers!}
	
\lstinputlisting[lastline=13]{centrecounty-slides.mod}

	\bit
	\item We need some way to provide the {\bf data} ({\bf sets} and {\bf parameter} values)
	\item The data go into a \myinline{.dat} file
	\item Separation of model and data is core to AMPL's philosophy
	\eit

\end{frame}


\begin{frame}[fragile]

	\frametitle{The Data File}
	
	\bit
	\item The \myinline{.dat} file specifies the items in each {\bf set} and the values of each {\bf parameter}
	\item Create a new file called \myinline{centrecounty.dat}:
	\eit
	
\begin{lstlisting}
set PROJECTS := park basketball rec pool;

param usage :=
	park		600
	basketball	100
	rec			300
	pool		500 ;
	
param cost :=
	park		50
	basketball	20
	rec			150
	pool		70 ;
	
param space :=
	park		8
	basketball	0
	rec			4
	pool		5 ;
\end{lstlisting}

\end{frame}

\begin{frame}[fragile]

	\frametitle{The Data File}

\begin{lstlisting}
set PROJECTS := park basketball rec pool;

param usage :=
	park		600
	basketball	100
	rec			300
	pool		500 ;
	
param cost :=
	park		50
	basketball	20
	rec			150
	pool		70 ;
	
param space :=
	park		8
	basketball	0
	rec			4
	pool		5 ;
\end{lstlisting}

	\bit
	\item Notice that:
		\bit
		\item Lines still end with semicolons
		\item Set elements can be strings (they can also be numbers)
		\item Declarations use the \myinline{:=} symbol
		\eit
	\eit

\end{frame}



\begin{frame}[fragile]

	\frametitle{Solving the Revised Model}

	\bit
	\item Now let's solve the revised model
	\eit
	
\begin{lstlisting}
ampl: reset;
ampl: model centrecounty.mod;
ampl: data centrecounty.dat;			<-- tells AMPL which data you want to use
ampl: solve;
CPLEX 12.8.0.0: optimal integer solution; objective 1200
0 MIP simplex iterations
0 branch-and-bound nodes
\end{lstlisting}

	\bit
	\item We can \myinline{display} decision variables, just like before, even if they are indexed by sets:
	\eit
	
\begin{lstlisting}
ampl: display Select;
Select [*] :=
basketball  1
      park  1
      pool  1
       rec  0
;
\end{lstlisting}

	\bit
	\item It's the same solution as before---not surprising, since it's the same model (and data)
	\eit

\end{frame}

\begin{frame}[fragile]

	\frametitle{Basketball Constraint}

	\bit
	\item Let's add the ``if basketball then park'' constraint back into the model
	\item Note that {\bf set} elements that are strings must be enclosed in single quotes
	\eit
	
\begin{lstlisting}
subj to IfBasketballThenPark: Select['basketball'] <= Select['park'];
\end{lstlisting}

	\bit
	\item (We already know that the solution will be the same, but let's solve it anyway)
	\eit
	
\begin{lstlisting}
ampl: reset;
ampl: model centrecounty.mod;
ampl: data centrecounty.dat;
ampl: solve;
CPLEX 12.8.0.0: optimal integer solution; objective 1200
0 MIP simplex iterations
0 branch-and-bound nodes
\end{lstlisting}

\end{frame}

\begin{frame}[fragile]

	\frametitle{A More Compact Data Syntax}
	
	\bit
	\item When several parameters have the same index set, we can combine them into one table:
	\eit
	
\begin{lstlisting}
set PROJECTS := park basketball rec pool;
	
param:		usage	cost	space :=
park		600		50		8
basketball	100		20		0
rec			300		150		4
pool		500		70		5 ;
\end{lstlisting}

	\bit
	\item White space (tabbing, etc.)\ doesn't matter, but nice alignment makes the table easier to read
	\pause
	\item Try the new data format and re-solve the model
	\item Did you get the same optimal solution?
	\eit

\end{frame}

\begin{frame}[fragile]

	\frametitle{Right-Hand Sides as Parameters}
	
	\bit
	\item The right-hand sides of our constraints are written as numbers
	\item It's good practice  to avoid ``hard-coding'' {\em any} numbers in the \myinline{.mod} file
	\item Instead, declare them as parameters:
	\eit
	
\begin{lstlisting}
param budget;				# available budget
param land_avail;			# available land
\end{lstlisting}

	\bit
	\item And replace the right-hand sides with those parameters:
	\eit
	
\begin{lstlisting}
subj to Budget: sum {j in PROJECTS} cost[j] * Select[j] <= budget;
	
subj to LandAvailable: sum {j in PROJECTS} space[j] * Select[j] <= land_avail;
\end{lstlisting}

	\bit
	\item And declare the new parameters in the \myinline{.dat} file:
	\eit
	
\begin{lstlisting}
param budget := 200;
param land_avail := 15;
\end{lstlisting}

\end{frame}

\begin{frame}[fragile]

	\frametitle{Right-Hand Sides as Parameters}
	
	\bit
	\item Let's solve the revised model:
	\eit
	
\begin{lstlisting}
ampl: reset;
ampl: model centrecounty.mod;
ampl: data centrecounty.dat;
ampl: solve;
CPLEX 12.8.0.0: optimal integer solution; objective 1200
0 MIP simplex iterations
0 branch-and-bound nodes
ampl: display Select;
Select [*] :=
basketball  1
      park  1
      pool  1
       rec  0
;
\end{lstlisting}

	\bit
	\item Yay.
	\eit
	
\end{frame}




\begin{frame}[fragile]

	\frametitle{``For All'' Constraints}
	
	\bit
	\item Suppose we want to add a constraint that says that the total spent on {\em each} project can be no more than 60:
		\[ c_j x_j \le 60 \qquad \forall j\in P \]
	\item To implement the ``for all,'' we {\em index the constraints}:
	\eit	
	
\begin{lstlisting}
subj to MaxSpendPerProject {j in PROJECTS}: cost[j] * Select[j] <= 60;
\end{lstlisting}
	
	\pause
	
	\bit
	\item Or better yet:
	\eit
	
\begin{lstlisting}
param max_spend;			# max spend per project
...
subj to MaxSpendPerProject {j in PROJECTS}: cost[j] * Select[j] <= max_spend;
\end{lstlisting}

	\bit
	\item And in \myinline{centrecounty.dat}:
	\eit

\begin{lstlisting}
param max_spend := 60;
\end{lstlisting}
	
\end{frame}

\begin{frame}[fragile]

	\frametitle{``For All'' Constraints}
	
	\bit
	\item Let's re-solve the model:
	\eit	
	
\begin{lstlisting}
ampl: reset;
ampl: model centrecounty.mod;
ampl: data centrecounty.dat;
ampl: solve;
CPLEX 12.8.0.0: optimal integer solution; objective 700
0 MIP simplex iterations
0 branch-and-bound nodes
ampl: display Select;
Select [*] :=
basketball  1
      park  1
      pool  0
       rec  0
;
\end{lstlisting}
	
\end{frame}




%%%%%%%%%%%%%%%%%%%%%
% ERRORS
%%%%%%%%%%%%%%%%%%%%%

\section{Things That Can Go Wrong}

\begin{frame}[fragile]

	\frametitle{Dangling Subscripts}
	
	\bit
	\item AMPL is {\em great} at recognizing dangling subscripts!
	\eit
	
\begin{lstlisting}
set PROJECTS;					# set of projects (P)

param usage {PROJECTS};			# usage for project j (u_j)
param cost {PROJECTS};			# cost for project j (c_j)
param space {PROJECTS};			# space for project j (s_j)

var Select {PROJECTS} binary; 	# select project j? (x_j)

maximize TotalUsage: usage[j] * Select[j];		<-- no sum!!!
	
...
\end{lstlisting}

	\bit
	\item Then:
	\eit

\begin{lstlisting}
ampl: model centrecounty.mod;

centrecounty.mod, line 9 (offset 281):
	j is not defined
context:  maximize TotalUsage:  >>> usage[j] <<<  * Select[j];
\end{lstlisting}

\end{frame}

\begin{frame}[fragile]

	\frametitle{Dangling Subscripts}
	
	\bit
	\item Or:
	\eit
	
\begin{lstlisting}
subj to MaxSpendPerProject: cost[j] * Select[j] <= max_spend;		<-- no for-all!!!
\end{lstlisting}


\begin{lstlisting}
ampl: model centrecounty.mod;

centrecounty.mod, line 23 (offset 722):
	j is not defined
context:  subj to MaxSpendPerProject:  >>> cost[j] <<<  * Select[j] <= max_spend;
\end{lstlisting}

\end{frame}


\begin{frame}[fragile]

	\frametitle{Same Index in Sum and For-All}
	
	\bit
	\item AMPL also complains if you use the same index in a sum and a for-all:
	\eit
	
\begin{lstlisting}
subj to MaxSpendPerProject {j in PROJECTS}: 
	sum {j in PROJECTS} cost[j] * Select[j] <= max_spend; <-- j in sum and for-all!!!
\end{lstlisting}


\begin{lstlisting}
ampl: model centrecounty.mod;

centrecounty.mod, line 24 (offset 747):
	syntax error
context:  sum {j  >>> in  <<< PROJECTS} cost[j] * Select[j] <= max_spend;
\end{lstlisting}

	\bit
	\item (The \myinline{syntax error} isn't that helpful, but the \myinline{>>> in  <<<} is)
	\eit

\end{frame}


\begin{frame}[fragile]

	\frametitle{Misspellings}
	
\begin{lstlisting}
maximize TotalUsage: sum {j in PROJECTS} usage[j] * Slect[j];	<-- oops!
\end{lstlisting}


\begin{lstlisting}
ampl: model centrecounty.mod;

centrecounty.mod, line 15 (offset 459):
	Slect is not defined
context:  maximize TotalUsage: sum {j in PROJECTS} usage[j] *  >>> Slect[j]; <<< 
\end{lstlisting}

	\pause 
	
	\bit
	\item Don't forget that AMPL is case-sensitive:
	\eit
	
\begin{lstlisting}
maximize TotalUsage: sum {j in PROJECTS} usage[j] * select[j];	<-- should be Select!
\end{lstlisting}


\begin{lstlisting}
ampl: model centrecounty.mod;

centrecounty.mod, line 15 (offset 459):
	select is not defined
context:  maximize TotalUsage: sum {j in PROJECTS} usage[j] *  >>> select[j]; <<< 
\end{lstlisting}

\end{frame}


\begin{frame}[fragile]

	\frametitle{Missing Data}
	
\begin{lstlisting}
param:		usage	cost	space :=
park		600		50		8
basketball	100		20		0
rec			300		150		4 ;				<-- no pool!
\end{lstlisting}


\begin{lstlisting}
ampl: model centrecounty.mod;
ampl: data centrecounty.dat;
ampl: solve;
Error executing "solve" command:
error processing objective TotalUsage:
	no value for usage['pool']
\end{lstlisting}

	\bit
	\item Notice that the error doesn't occur until you \myinline{solve}
	\eit
	
\end{frame}


\begin{frame}[fragile]

	\frametitle{Missing Semicolon}
	
\begin{lstlisting}
maximize TotalUsage: sum {j in PROJECTS} usage[j] * Select[j]	<-- no semicolon!
\end{lstlisting}


\begin{lstlisting}
ampl: model centrecounty.mod;

centrecounty.mod, line 17 (offset 471):
	syntax error
context:  >>> subj  <<< to Budget: sum {j in PROJECTS} cost[j] * Select[j] <= budget;
\end{lstlisting}

	\bit
	\item Notice that the error actually points to the {\em next} line, because that's when AMPL realizes something is wrong
	\eit
	
\end{frame}

\begin{frame}[fragile]

	\frametitle{Missing Semicolon}

	\bit
	\item If you forget the semicolon on the command line, AMPL just prompts you for it, like a parent waiting for a ``please'':
	\eit
	
\begin{lstlisting}
ampl: model centrecounty.mod;
ampl: data centrecounty.dat;
ampl: solve
ampl? 
\end{lstlisting}

	\bit
	\item Then just type the \myinline{;} and you are forgiven:
	\eit
	
\begin{lstlisting}
ampl? ;
CPLEX 12.8.0.0: optimal integer solution; objective 700
0 MIP simplex iterations
0 branch-and-bound nodes
\end{lstlisting}

\end{frame}



%%%%%%%%%%%%%%%%%%%%%
% WYNDOR GLASS
%%%%%%%%%%%%%%%%%%%%%

\section{Wyndor Glass Model}

\begin{frame}[fragile]

	\frametitle{Remember Wyndor Glass?}

	\beq \begin{array}{lrcrcl}
		\maximize & 3x_1 & + & 5x_2 & & \cr
		\st & x_1 & & & \le & 4 \cr
		& & & 2x_2 & \le & 12 \cr
		& 3x_1 & + & 2x_2 & \le & 18 \cr
		& x_1 & , & x_2 & \ge & 0 
	\end{array} \eeq
	
	\bit
	\item 2 products, 3 plants
	\item Objective maximizes total profit
	\item Constraints enforce plant capacities
	\eit
	

\end{frame}

\begin{frame}[fragile]

	\frametitle{Wyndor Glass in Algebraic Notation}

	\bit
	\item Sets:
		\bit
		\item $I$ = set of plants
		\item $J$ = set of products
		\eit
	\item Parameters:
		\bit
		\item $r_j$ = profit per batch of product $j$ sold
		\item $b_i$ = available hours at plant $i$
		\item $a_{ij}$ = production time per batch of product $j$ made at plant $i$
		\eit
	\item Decision variables:
		\bit
		\item $x_j$ = number of batches of product $j$ produced
		\eit
	\eit
	
	{\footnotesize
	\begin{alignat*}{2}
	\text{maximize} \quad & \sum_{j\in J} r_jx_j && \\
	\text{subject to} \quad & \sum_{j\in J} a_{ij}x_j \le b_i &\quad& \forall i\in I \\
		& x_j \ge 0 && \forall j\in J
	\end{alignat*}
	}
	
\end{frame}


\begin{frame}[fragile]

	\frametitle{{\tt wyndor.mod}}

	{\footnotesize
	\begin{alignat*}{2}
	\text{maximize} \quad & \sum_{j\in J} r_jx_j && \\
	\text{subject to} \quad & \sum_{j\in J} a_{ij}x_j \le b_i &\quad& \forall i\in I \\
		& x_j \ge 0 && \forall j\in J
	\end{alignat*}
	}
	
	\vspace{-1em}
	
\begin{lstlisting}
set PLANTS;						# set of plants (I)
set PRODUCTS;					# set of products (J)

param profit {PRODUCTS};		# profit per unit of product j (r_j)
param avail_hours {PLANTS};		# available hours at plant i (b_i)
param hours_per_batch {PLANTS,PRODUCTS};
								# num hours at plant i per batch of product j (a_ij)
							
var Produce {PRODUCTS} >= 0;	# num batches of product j to produce (x_j)

maximize TotalProfit: sum {j in PRODUCTS} profit[j] * Produce[j];

subj to PlantCapacity {i in PLANTS}: 
	sum {j in PRODUCTS} hours_per_batch[i,j] * Produce[j] <= avail_hours[i];
\end{lstlisting}

\end{frame}

\begin{frame}[fragile]

	\frametitle{{\tt wyndor.dat}}

		\centerline{\scriptsize
		\begin{tabular}{c|cc|c}
		& \multicolumn{2}{c|}{Production Time} & \\
		Plant & Product 1 & Product 2 & Available Hours \\ \hline
		1 & 1 & 0 & 4 \\
		2 & 0 & 2 & 12 \\
		3 & 3 & 2 & 18 \\ \hline
		Profit per Batch (x1000) & \$3 & \$5 & \\ \hline
		\end{tabular}
		}
	
	\vspace{1em}
	
\begin{lstlisting}
set PLANTS := 1 2 3;		
set PRODUCTS := 1 2;

param profit :=
	1	3
	2	5 ;
	
param avail_hours :=
	1	4
	2	12
	3	18 ;
	
param hours_per_batch :		<-- syntax for 2-index parameters in .dat file
		1	2	:=
	1	1	0
	2	0	2
	3	3	2 ;
\end{lstlisting}

\end{frame}

\begin{frame}[fragile]

	\frametitle{Solving the Model}
	
\begin{lstlisting}
ampl: model wyndor.mod;
ampl: data wyndor.dat;
ampl: solve;
CPLEX 12.8.0.0: optimal solution; objective 36
1 dual simplex iterations (0 in phase I)
ampl: display Produce;
Produce [*] :=
1  2
2  6
;
\end{lstlisting}

	\bit
	\item Our old friend $x^* = (2,6)$, $Z^* = 36$
	\eit

\end{frame}



%%%%%%%%%%%%%%%%%%%%%
% OTHER STUFF
%%%%%%%%%%%%%%%%%%%%%

\section{Other Things to Know}

\begin{frame}[fragile]

	\frametitle{Solvers}
	
	\bit
	\item AMPL can interface with {\em lots} of different {\bf solvers}
	\item The solver is the code that actually does the optimization
	\item The AMPL input to the solver (\myinline{.mod} and \myinline{.dat} files) and the AMPL output from the solver (e.g., \myinline{display Select;}) are the same, regardless of solver
	\item For most purposes, CPLEX or Gurobi are sufficient
		\bit
		\item They can solve LPs, IPs, and other types of problems
		\item But you need a license
		\eit
	\eit
	
\end{frame}


\begin{frame}[fragile]

	\frametitle{Sets of Numbers}
	
	\bit
	\item AMPL makes it easy to declare sets of numbers:
		\bit
		\item \myinline{set PERIODS := 1..20} $\implies$ \myinline{PERIODS = \{1, 2, ..., 20\}}
		\item \myinline{set PERIODS := 5..50 by 5} $\implies$ \myinline{PERIODS = \{5, 10, ..., 50\}}
		\eit
	\item A common pattern is to declare the set {\em size} in the \myinline{.mod} file, declare the {\em set} based on the size also in the \myinline{.mod} file, and specify the size in the \myinline{.dat} file:
	\eit

\myinline{.mod}:	
\begin{lstlisting}
param T;

set PERIODS := 1..T;
\end{lstlisting}

\myinline{.dat}:	
\begin{lstlisting}
param T := 20;
\end{lstlisting}

	\bit
	\item {\bf Note:} ``\myinline{..}'' notation can only be used in \myinline{.mod} files, not \myinline{.dat}
	\eit
	
\end{frame}

\begin{frame}[fragile]

	\frametitle{Sets of Numbers}
	
	\bit
	\item You can also use sets of numbers without explicitly declaring them as sets
	\eit

\myinline{.mod}:	
\begin{lstlisting}
param cost {1..T};
var Produce {1..T};

minimize TotalCost: sum {1..T} cost[t] * Produce[t];
\end{lstlisting}

\myinline{.dat}:	
\begin{lstlisting}
param T := 20;
\end{lstlisting}

	\bit
	\item (But I find it cleaner to declare the sets)
	\eit
	
\end{frame}


\begin{frame}[fragile]

	\frametitle{Parameter Conditions}
	
	\bit
	\item You can impose conditions on parameters in the \myinline{.mod} file
	\item These are not constraints! They are just validation rules
	\eit

\myinline{.mod}:	
\begin{lstlisting}
param budget >= 0;
param land_avail > 0;
param is_open binary;
param num_staff integer;
param staffing_level >= 7;
\end{lstlisting}

\myinline{.dat}:	
\begin{lstlisting}
param budget := 200;
param land_avail := -15;
\end{lstlisting}

\begin{lstlisting}
ampl: model centrecounty.mod;
ampl: data centrecounty.dat;
ampl: solve;
Error executing "solve" command:
error processing param land_avail:
	failed check: param land_avail = -15
		is not > 0;
\end{lstlisting}

\end{frame}

\begin{frame}[fragile]

	\frametitle{Set Operations}
	
	\bit
	\item AMPL provides operators to work with sets:
		\bit
		\item \myinline{union} means $A \cup B$
		\item \myinline{inter} means $A \cap B$
		\item \myinline{diff} means $A \setminus B$
		\item \myinline{symdiff} means symmetric difference: in $A$ or $B$ but not both
		\item \myinline{cross} means cross product (Cartesian product): pairs $(a,b)$ with $a\in A$, $b\in B$
		\eit
	\eit

\begin{lstlisting}
set SUPPLIERS;
set FACTORIES;
set NODES = SUPPLIERS union FACTORIES;
set EXTERNAL_SUPPLIERS = SUPPLIERS diff FACTORIES;
set LINKS = SUPPLIERS cross FACTORIES;
\end{lstlisting}

\end{frame}

\begin{frame}[fragile]

	\frametitle{{\tt .run} Files}
	
	\bit
	\item If you find yourself repeating the same commands over and over again...
	\item ...use a \myinline{.run} file!
	\item A \myinline{.run} file is a script consisting of standard AMPL commands
	\eit

\myinline{wyndor.run}:
\begin{lstlisting}
reset;
model wyndor.mod;
data wyndor.dat;
option solver cplex;
solve;
display Produce;
\end{lstlisting}

	\bit
	\item Then, run your file from the command line:
	\eit

\begin{lstlisting}
ampl: include wyndor.run;
\end{lstlisting}

\end{frame}


\begin{frame}[fragile]

	\frametitle{{\tt .run} Files}
	
	\bit
	\item You can also use \myinline{for}, \myinline{let}, \myinline{printf}, etc.
	\eit

%\myinline{wyndor.run}:
\begin{lstlisting}
reset;
model wyndor.mod;
data wyndor.dat;
option solver cplex;

set HOURS_TO_TEST := avail_hours[3] .. avail_hours[3] + 10 by 2;
for {h in HOURS_TO_TEST} {
	let avail_hours[3] := h;
	solve;
	printf "avail_hours[3] = %2d ==> optimal profit = %6.2f\n", h, TotalProfit;
}
\end{lstlisting}

\begin{lstlisting}
ampl: include wyndor.run;
CPLEX 12.8.0.0: optimal solution; objective 36
1 dual simplex iterations (0 in phase I)
avail_hours[3] = 18 ==> optimal profit =  36.00
<-- suppressing further CPLEX output... -->
avail_hours[3] = 20 ==> optimal profit =  38.00
avail_hours[3] = 22 ==> optimal profit =  40.00
avail_hours[3] = 24 ==> optimal profit =  42.00
avail_hours[3] = 26 ==> optimal profit =  42.00
avail_hours[3] = 28 ==> optimal profit =  42.00
\end{lstlisting}

\end{frame}


\begin{frame}[fragile]

	\frametitle{{\tt .run} Files}
	
	\bit
	\item \myinline{if-then-else} conditions, too:
	\eit

%\myinline{wyndor.run}:
\begin{lstlisting}
...
for {h in HOURS_TO_TEST} {
	let avail_hours[3] := h;
	solve;
	if Produce[1] >= 3 then 
		printf "avail_hours[3] = %2d ==> producing >= 3 batches of product 1\n", h;
	else 
		printf "avail_hours[3] = %2d ==> producing <3 batches of product 1\n", h;
}
\end{lstlisting}

\begin{lstlisting}
CPLEX 12.8.0.0: optimal solution; objective 36
1 dual simplex iterations (0 in phase I)
avail_hours[3] = 18 ==> producing <3 batches of product 1
<-- suppressing further CPLEX output... -->
avail_hours[3] = 20 ==> producing <3 batches of product 1
avail_hours[3] = 22 ==> producing >= 3 batches of product 1
avail_hours[3] = 24 ==> producing >= 3 batches of product 1
avail_hours[3] = 26 ==> producing >= 3 batches of product 1
avail_hours[3] = 28 ==> producing >= 3 batches of product 1
\end{lstlisting}

\end{frame}




%%%%%%%%%%%%%%%%%%%%%
% YOUR TURN
%%%%%%%%%%%%%%%%%%%%%

\section{Your Turn}


\begin{frame}[fragile]

	\frametitle{Your Turn: Fixed-Charge Problem}

	\bit
	\item $N$ products we can manufacture
	\item Product $j$:
		\bit
		\item Incurs a fixed cost $K_j$ if we manufacture it
		\item Incurs a variable cost $C_j$ per unit manufactured
		\item Earns a profit $p_j$ per unit sold
		\item Has a demand potential $d_j$
		\eit
	\item $M$ raw materials
	\item Each unit of product $j$ manufactured requires $a_{ij}$ units of raw material $i$
	\item $b_i$ units of raw material $i$ are available
	\item Goal: select  product mix to maximize net profit
	\eit
	
	
\end{frame}
	
	
	
\begin{frame}[fragile]

	\frametitle{Your Turn: Fixed-Charge Problem}
	
	{\footnotesize
	\begin{alignat*}{2}
	\text{maximize} \quad & \sum_{j=1}^N p_jx_j - \sum_{j=1}^N (K_j\delta_j + C_jx_j) && \\
	\text{subject to} \quad & \sum_{j=1}^N a_{ij}x_j \le b_i && \forall i=1,\ldots,M \\
		& x_j \le d_j\delta_j && \forall j=1,\ldots,N \\
		& x_j \ge 0 && \forall j=1,\ldots,N \\
		& \delta_j \in \{0,1\} && \forall j=1,\ldots,N
	\end{alignat*}
	}

	\vspace{1em}
	
	{\small {\em Source}: Ravindran, Griffin, and Prabhu, {\em Service Systems Engineering and Management}, CRC Press, 2018.}
	
\end{frame}

\begin{frame}[fragile]

	\frametitle{Your Turn: Fixed-Charge Problem}

	\bit
	\item Use the following data:
	\eit
	
		\vspace{1em}
		
		\centerline{\scriptsize
		\begin{tabular}{c|cccc|c}
		& \multicolumn{4}{c|}{Resources Needed ($a_{ij}$)} & \\
		Resource & Prod.\ 1 & Prod.\ 2 & Prod.\ 3 & Prod.\ 4 & \# Available ($b_i$) \\ \hline
		1 & 4 & 1 & 0 & 2 & 400 \\
		2 & 0 & 2 & 3 & 2 & 600 \\
		3 & 1 & 1 & 1 & 1 & 300 \\ \hline
		Fixed cost ($K_j$) & 100 & 150 & 50 & 100 \\ 
		Variable cost ($C_j$) & 10 & 10 & 40 & 30 \\
		Sales price ($p_j$) & 22 & 30 & 18 & 45 \\
		Sales potential ($d_j$) & 100 & 75 & 140 & 60 \\ \hline
		\end{tabular}
		}
	
\end{frame}

\begin{frame}
	\frametitle{This Slide Intentionally Left Blank...}
\end{frame}

\begin{frame}[fragile]

	\frametitle{My {\tt .mod} File}
	
	
\begin{lstlisting}
param num_resources;					# number of resources (M)
param num_products;						# number of products (N)

set RESOURCES := 1..num_resources;		# set of resources
set PRODUCTS := 1..num_products;		# set of products

param fixed_cost {PRODUCTS};			# fixed cost (K_j)
param var_cost {PRODUCTS};				# variable cost (C_j)
param sales_price {PRODUCTS};			# sales price (p_j)
param sales_potential {PRODUCTS};		# sales potential (d_j)
param avail {RESOURCES};				# resource availability (b_i)
param needs {RESOURCES, PRODUCTS};		# units of resource needed for product (a_ij)

var ProduceAmt {PRODUCTS} >= 0;			# num units of product to produce (x_j)
var Produce {PRODUCTS} binary;			# 0/1 if we produce/don't produce (delta_j)

maximize Profit:
	sum {j in PRODUCTS} sales_price[j] * ProduceAmt[j] 
	- sum {j in PRODUCTS} (fixed_cost[j] * Produce[j] + var_cost[j] * ProduceAmt[j]);
	
subject to Supply {i in RESOURCES}:
	sum {j in PRODUCTS} needs[i,j] * ProduceAmt[j] <= avail[i];
	
subject to LinkingAndSalesPotential {j in PRODUCTS}:
	ProduceAmt[j] <= sales_potential[j] * Produce[j];
\end{lstlisting}

\end{frame}

\begin{frame}[fragile]

	\frametitle{My {\tt .dat} File}
	
	
\begin{lstlisting}
param num_resources := 3;
param num_products := 4;

param:	fixed_cost	var_cost	sales_price	sales_potential :=
	1	100			10			22			100
	2	150			10			30			75
	3	50			40			18			140
	4	100			30			45			60 ;
	
param avail :=
	1	400
	2	600
	3	300 ;
	
param needs :
		1	2	3	4	:=
	1	4	1	0	2
	2	0	2	3	2
	3	1	1	1	1 ;
\end{lstlisting}

\end{frame}

\begin{frame}[fragile]

	\frametitle{My AMPL Transcript}
	
	
\begin{lstlisting}
ampl: model fixedcharge.mod;
ampl: data fixedcharge.dat;
ampl: solve;
CPLEX 12.8.0.0: optimal integer solution; objective 2665
0 MIP simplex iterations
0 branch-and-bound nodes
absmipgap = 4.54747e-13, relmipgap = 1.70637e-16
ampl: display Produce;
Produce [*] :=
1  1
2  1
3  0
4  1
;

ampl: display ProduceAmt;
ProduceAmt [*] :=
1  51.25
2  75
3   0
4  60
;
\end{lstlisting}

	\bit
	\item So: We produce products 1, 2, and 4, with $x^* = (51.25,75,0,60)$ and an optimal profit of 2665
	\eit

\end{frame}



\end{document}
