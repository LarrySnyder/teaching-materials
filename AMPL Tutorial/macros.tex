%%%%%%%%%%%%%%%%%%%%%%%%%%%%%%%%%%%%%%%%%%%%%%%%%%%%%%%%%%%%%%%%%%%%%%%%%%%%%%%%%%%%%%%%%%%%%%%%%
%
% Headers and macros for slides for ISE 240 (Intro to Operations Research)
% Based on Hiller and Lieberman textbook, "Introduction to Operations Research" 
% Slides developed by Prof. Larry Snyder, Lehigh University
%
% Please do not distribute LaTeX source code for slides without prior permission from the author
% Contact him at larry.snyder@lehigh.edu or http://coral.ie.lehigh.edu/~larry
%
%%%%%%%%%%%%%%%%%%%%%%%%%%%%%%%%%%%%%%%%%%%%%%%%%%%%%%%%%%%%%%%%%%%%%%%%%%%%%%%%%%%%%%%%%%%%%%%%%


\documentclass{beamer} 
%\documentclass[handout]{beamer}

\usepackage{pgf,pgfarrows,pgfnodes,pgfautomata,pgfheaps,pgfshade}
\usepackage{amsmath,amssymb}
\usepackage[latin1]{inputenc}
\usepackage{colortbl}
\usepackage[english]{babel}
\usepackage[normalem]{ulem} % for strikethrough font

\usepackage{tikz}
\usetikzlibrary{snakes}
\usetikzlibrary{decorations}
\usetikzlibrary{calc}
\usepgflibrary{shapes.geometric}
\usetikzlibrary{intersections}
\usetikzlibrary{positioning}

\usepackage{pgfplots}
\usepgfplotslibrary{fillbetween}


\usepackage{graphicx}
\usepackage[breakable]{tcolorbox}
\usepackage{alltt}

\usepackage{inconsolata} % fixed-width font for tt

% listings package
\usepackage{listings}
\lstset{% set up for AMPL
	basicstyle=\tiny\ttfamily,
	keywordstyle=\color{purple},
	commentstyle=\color{ForestGreen},
	backgroundcolor=\color{blue!5},
	frame=lines,
	morecomment=[l]{\#},
	morecomment=[l][\color{BrickRed}]{<--}, % not really a comment, just for pointing things out on the slides
	morekeywords={param, var, set, maximize, minimize, sum, subject, subj, to, binary, model, data, solve, union, inter, diff, symdiff, cross, reset, option, display, include, printf, let, for, if, then, else},
	sensitive=true,
	tabsize=4,
}
\newcommand{\myinline}[1]{\lstinline[basicstyle=\small\ttfamily]!#1!}

\usepackage{beamerthemetree} 			% choose beamer theme

\input{dvipsnam.def}				% color definitions

\beamertemplatenavigationsymbolsempty		% turn off navigation symbols

\usefoottemplate{\tinycolouredline{white}{\insertshorttitle\hfill\insertframenumber}}
						% insert footer: date and page number

\beamerboxesdeclarecolorscheme{allblue}{structure!75!black}{structure!75!black}
\beamerboxesdeclarecolorscheme{twotone}{structure!75!black}{structure!10!averagebackgroundcolor}
						% define color schemes for beamer boxes
						
\setcounter{MaxMatrixCols}{20}			% reset maximum # of columns in matrix environment


% internal comments -- to suppress, comment out first line below and un-comment second line
%\newcommand{\comment}[1]{\textcolor{red}{\bf [#1]}}
%\newcommand{\comment}[1]{}


% environment for code examples
\newenvironment{mycode}{\vspace{0em}\begin{tcolorbox}[breakable,boxrule=0.25mm]\begin{alltt}}{\end{alltt}\end{tcolorbox}\vspace{0em}}



% environments
\newcommand{\beq}{\begin{equation*}}
\newcommand{\eeq}{\end{equation*}}
\newcommand{\bit}{\begin{itemize}}
\newcommand{\eit}{\end{itemize}}
\newcommand{\ben}{\begin{enumerate}}
\newcommand{\een}{\end{enumerate}}

% LP text
\newcommand{\minimize}{\text{minimize}}
\newcommand{\maximize}{\text{maximize}}
\newcommand{\st}{\text{subject to}}
\newcommand{\shortmin}{\text{min}}
\newcommand{\shortmax}{\text{max}}
\newcommand{\shortst}{\text{s.t.}}
\newcommand{\primal}{\ensuremath{(\text{P})}}
\newcommand{\dual}{\ensuremath{(\text{D})}}

% math shortcuts
\newcommand{\smfrac}[2]{\textstyle{\frac{#1}{#2}}}
\newcommand{\artvar}[1]{\bar{x}_{#1}}
\newcommand{\matname}[1]{\mathbf{#1}}
\newcommand{\mymat}[1]{\left[\begin{matrix} #1 \end{matrix}\right]}
\newcommand{\matA}{\matname{A}}
\newcommand{\matAinv}{\matA^{-1}}
\newcommand{\matx}{\matname{x}}
\newcommand{\matb}{\matname{b}}
\newcommand{\matI}{\matname{I}}
\newcommand{\maty}{\matname{y}}
\newcommand{\matnull}{\matname{0}}
\newcommand{\matc}{\matname{c}}
\newcommand{\matB}{\matname{B}}
\newcommand{\matBinv}{\matB^{-1}}
\newcommand{\pijn}[1]{p_{ij}^{(#1)}}
\newcommand{\matP}{\matname{P}}
\newcommand{\matPn}[1]{\matP^{(#1)}}
\newcommand{\liminfty}{\lim_{n\rightarrow\infty}}

% variance
\newcommand{\var}{\text{Var}}
\newcommand{\cov}{\text{Cov}}

% font styles
\newcommand{\vocab}[1]{\textcolor{Blue}{\bf #1}}	% for vocabulary words
\newcommand{\basic}[1]{\textcolor{Red}{#1}}		% for basic variables

% Beamer boxes
\newcommand{\boxtext}[1]{\begin{beamerboxesrounded}[scheme=allblue]{}{\textcolor{White}{#1}}\end{beamerboxesrounded}}
\newcommand{\boxtexttitle}[2]{\begin{beamerboxesrounded}[scheme=twotone]{\textcolor{White}{#1}}{#2}\end{beamerboxesrounded}}
